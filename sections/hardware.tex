\providecommand{\main}{..}  
% *Modification: redefine path location, must go before \documentclass
\documentclass[../computer-history.tex]{subfiles}

\begin{document}
From here is where the conflict with INTEL begins especially with the latter’s creation of a microprocessor in 1974. AMD then followed in 1975 after reverse engineering the INTEL 8080 (Intel’s second microprocessor) the AM9080. In 1978 when INTEL entered into a contract with IBM who, due to the sheer volume of processors required, requested that this contract would be completed through the use of contractors, with AMD among those chosen. This secured AMD’s entry into the x86 processor market and created a technology exchange agreement with Intel so the smaller company could produce the product at IBM’s desired specification. Thus AMD began production of the AM286, a copy of the INTEL80826, which held the advantage of much higher clock speeds (Intel resting at around 12.5 MHz and AMD at 20 MHz). This, in addition to the 1975 AM9080, would start a trend between AMD and Intel where Intel would produce a device and AMD, through their agreement, would take the design and create a clone which would have a much higher clock speed. The next notable release was the AM486 and the AM5x86, released in 1994 and 1985 respectively. Both devices ran the same architecture but the AM5x86 ran much faster at 133 MHz (or potentially 150 MHz) as compared to the AM486’s speed of 120 MHz. These devices were the final Intel clones and marked the beginning of AMD creating CPUs of their own design and a separation from Intel.

\onecolumn
\\\textbf{1982 INTEL 8086 clone}\\ \\
\begin{tabular}{c|c}
    Release Date & 1982 \\
    Architecture & 16-bit \\
    Data Bus & 16-bit \\
    Address Bus & 20-bit 
    Maximum Memory Support & 1 MB \\
    L1 Cache & None \\
    L2 Cache & None \\
    Frequency & 4 - 10 MHz \\
    FSB & 4 - 10 MHz \\
    FPU & 8087 (Sold Separate) \\
    SIMD & None  \\
    Fab & 3000 nm \\
    Transistor Count & 29,000 \\
    Power Consumption & N/A \\
    Voltage & 5 V \\
    Die Area & 33 mm² \\
    Socket &40 pins
\end{tabular}

\ \\ \\
\textbf{AMD Bulldozer Zambezi} \\ \\
\begin{tabular}{c|c}
    Code Name & Zambezi \\
    Date & October 2011 \\
    Architecture & 64-bit \\
    Data Bus & 64-bit \\
    Address Bus & 64-bit \\
    Maximum Memory Support & 1 TB \\
    L1 Cache & 64 KB + (2 x 16 KB) \\
    L2 Cache & 2 MB (Full Speed) \\
    L3 Cache & 8 MB \\
    Clock Speed & 2.8 - 4.2 GHz (4.3 GHz Turbo) \\
    Memory Controller & Dual-Channel DDR3-1866 \\
    HyperTransport & 2600 MHz \\
    Core Count & 4, 6, 8 \\
    SIMD & MMX, SSE, SSE2, SSE3, SSSE3, SSE4a, SSE4.1/4.2, AVX \\
    Instructions & AES, FMA4, XOP \\
    Fab & 32 nm \\
    Transistor Count & N/A \\
    Power Consumption & 95 - 125 W \\
    Voltage & 0.95 - 1.4125 V \\
    Die Area & 316 mm² \\
    Socket & AM3+
\end{tabular}

\ \\ \\
\textbf{AMD Ryzen 9 7950X} \\ \\
\begin{tabular}{c|c}
    Core name & Raphael \\
    Release date & September 27, 2022 \\
    Generation & 5th \\
    Architecture & Zen 4 \\
    Frequency & 4.5 GHz \\
    L1 Cache & 1MB \\
    L2 Cache & 16MB \\
    L3 Cache & 64MB \\
    Cores & 16 \\
    Threads & 32 \\
    Unlocked & Yes \\
    Socket & AM5 \\
    Thermal Solution & Not included \\
    Thermal Design Power & (TDP)	170 W \\
    Max. Temperature & 95 °C \\
    Word Size & 64 bit \\
    TSMC FinFET process (Lithography) & 5 nm \\
    Transistor count & 12600 million \\
\end{tabular}
\two coulmn
%********************************************************************************
%      Bibliography
%********************************************************************************
\biblio
\end{document}
